\chapter{Graph Theory}%
\label{chap:graph-theory}

\section{Vocabulary}%
\label{sec:vocabulary}

\begin{definition}
  A \textbf{vertex} is a dot or circle, often labelled with a letter
  or short name. An edge is a line or arc which connects two
  \textbf{vertices}. A \textbf{loop} is a special kind of edge that
  connects a vertex to itself.
\end{definition}

\begin{definition}
  A \textbf{graph} is a pair of sets, a set of \textbf{vertices}, and
  a set of \textbf{edges}.
\end{definition}

\begin{definition}
  The \textbf{degree} of a vertex is the number of edges meeting at
  that vertex.
\end{definition}

\begin{definition}
  A \textbf{path} is an ordered sequence of edges.
\end{definition}
\begin{note}
  For brevity we sometimes denote a path as a sequence of vertices.
\end{note}

\begin{definition}
  A \textbf{circuit} is a path that begins and ends at the same
  vertex.
\end{definition}

\begin{definition}
  A graph is \textbf{connected} if there is a path from any vertex to
  any other vertex.
\end{definition}

\begin{definition}
  A \textbf{weighted graph} is a graph where each edge has a numerical
  value associated with it. The weights often correspond with
  distances, travel time, or travel cost.
\end{definition}

\section{Dijkstra's Algorithm (Shortest Path)}%
\label{sec:dijkstras-algorithm}

\begin{algorithm}[Dijkstra's Algorithm]
  \begin{enumerate}
  \item Mark the ending vertex with a distance of zero. Designate this
    vertex as current.

  \item Find all vertices leading to the current vertex. Calculate
    their distances to the end. Since we already know the distance the
    current vertex is from the end, this will just require adding the
    most recent edge. Don't record this distance if it is longer than
    a previously recorded distance.

  \item Mark the current vertex as visited. We will never look at this
    vertex again.

  \item Mark the vertex with the smallest distance as current, and
    repeat from step 2.
  \end{enumerate}
\end{algorithm}

\section{Eulerian Circuits and the Chinese Postman Problem}%
\label{sec:eulerian-circuits}

\subsection{Fleury's Algorithm}%
\label{sub:Fleury's Algorithm}

\begin{algorithm}[Chinese Postman --- Fleury]
  \begin{enumerate}
  \item Start at any vertex if finding an Euler circuit. If finding an
    Euler path, start at one of the two vertices with odd degree.
  \item Choose any edge leaving your current vertex, provided deleting
    that edge will not separate the graph into two disconnected sets
    of edges.
  \item Add that edge to your circuit, and delete it from the graph.
  \item Continue until you're done.
  \end{enumerate}
\end{algorithm}

\section{Hamiltonian Circuits and the Traveling Salesman Problem}%
\label{sec:hamiltonian-circuits}

\subsection{Brute Force}%
\label{sub:brute-force}

\begin{algorithm}[Brute Force]
  \begin{enumerate}
  \item Create a list of all possible circuits and their costs.
  \item Choose the circuit with the least cost.
  \end{enumerate}
\end{algorithm}

\subsection{Nearest Neighbor Algorithm (NNA)}%
\label{sub:nearest-neighbor}

\begin{algorithm}[Nearest Neighbor Algorithm]
  \begin{enumerate}
  \item Select a starting point.
  \item Move to the nearest unvisited vertex (via the edge with
    smallest weight).
  \item Repeat until the circuit is complete.
  \end{enumerate}
\end{algorithm}

\subsection{Repeated Nearest Neighbor Algorithm (RNNA)}%
\label{sub:repeated-nearest-neighbor}

\begin{algorithm}[Repeated Nearest Neighbor Algorithm]
  \begin{enumerate}
  \item Do the Nearest Neighbor Algorithm starting at each vertex.
  \item Choose the circuit produced with minimal total weight.
  \end{enumerate}
\end{algorithm}

\subsection{Sorted Edges Algorithm (SEA)}%
\label{sub:sorted-edges}

\begin{algorithm}[Sorted Edges Algorithm]
  \begin{enumerate}
  \item Select the cheapest unused edge in the graph.
  \item Repeat step 1, adding the cheapest unused edge to the circuit,
    unless:
    \begin{enumerate}
    \item adding the edge would create a circuit that doesn't contain
      all vertices, or
    \item adding the edge would give a vertex degree 3.
    \end{enumerate}
  \item Repeat until a circuit containing all vertices is formed.
  \end{enumerate}
\end{algorithm}

\section{Spanning Trees and Kruskal's Algorithm}%
\label{sec:spanning-trees}



%%% Local Variables:
%%% mode: latex
%%% TeX-master: "Notes"
%%% End:
