\chapter{Graph Theory}%
\label{chap:graph-theory}

\section{Vocabulary}%
\label{sec:vocabulary}

\begin{definition}
  A \textbf{vertex} is a dot or circle, often labeled with a letter
  or short name. An \textbf{edge} is a line or arc which connects two
  \textbf{vertices}. A \textbf{loop} is a special kind of edge that
  connects a vertex to itself.
\end{definition}

\begin{definition}
  A \textbf{graph} is a pair of sets, a set of \textbf{vertices}, and
  a set of \textbf{edges}.
\end{definition}

\begin{figure}[h!]
  \centering
  \includestandalone{figures/example-graph}
  \caption{Example graph}%
  \label{fig:example-graph}
\end{figure}

\newpage

\begin{definition}
  The \textbf{degree} of a vertex is the number of edges meeting at
  that vertex.
\end{definition}

\begin{example}
  For the graph in Figure~\ref{fig:example-graph} and
  Figure~\ref{fig:example-path-circuit} below, the vertices have the
  following degrees:
  \begin{align*}
    \deg(A) &= 3 & \deg(D) &= 4 \\
    \deg(B) &= 3 & \deg(E) &= 3 \\
    \deg(C) &= 3 & \deg(F) &= 4 \\
  \end{align*}
\end{example}

\begin{definition}
  A \textbf{path} is a sequence of edges.
\end{definition}
\begin{figure}[h!]
  \centering
  \includestandalone{figures/example-path-circuit}
  \caption{Example path in red and example circuit in blue}%
  \label{fig:example-path-circuit}
\end{figure}

\begin{note}
  For brevity we sometimes denote a path as a sequence of vertices.
  That is, we will often denote the path in
  Figure~\ref{fig:example-path-circuit} as \(P=A,B,C,F\)
\end{note}

\begin{definition}
  A \textbf{circuit} is a path that begins and ends at the same
  vertex.
\end{definition}

\newpage

\begin{definition}
  A graph is \textbf{connected} if there is a path from any vertex to
  any other vertex.
\end{definition}

\begin{definition}
  A \textbf{weighted graph} is a graph where each edge has a numerical
  value associated with it. The weights often correspond with
  distances, travel time, or travel cost.
\end{definition}

\begin{definition}
  A \textbf{graph algorithm} is like a function, it takes a graph as
  input and maybe a starting point and outputs either a new graph, or
  a path or a circuit. An algorithm is essentially a sequence of steps
  or operations to perform on a graph to obtain a desired result.

  Some graph algorithms are \textbf{efficient} meaning that they are
  guaranteed to finish in a short amount of time. Normal execution
  times could be anywhere from a few milliseconds to a few hours. Some
  graph algorithms are \textbf{optimal} meaning they are guaranteed to
  find the best possible solution.
\end{definition}

\section{Dijkstra's Algorithm (Shortest Path)}%
\label{sec:dijkstras-algorithm}

One common usage for graphs is pathfinding. In this case the vertices
represent locations, \eg{} cities, and the edges represent roads, rail
lines, or perhaps connecting flights. Most often these graphs will be
weighted where the weights represent distance, travel time, or travel
cost.

\begin{algorithm}[Dijkstra's Algorithm]
  \begin{enumerate}
  \item Mark the ending vertex with a distance of zero. Designate this
    vertex as current.

  \item Find all vertices leading to the current vertex. Calculate
    their distances to the end. Since we already know the distance the
    current vertex is from the end, this will just require adding the
    most recent edge. Don't record this distance if it is longer than
    a previously recorded distance.

  \item Mark the current vertex as visited. We will never look at this
    vertex again.

  \item Mark the vertex with the smallest distance as current, and
    repeat from step 2.
  \end{enumerate}
\end{algorithm}

\begin{note}
  Dijkstra's algorithm is both \emph{efficient} and \emph{optimal}. It
  is so efficient (fast) that it is used by GPS navigation systems to
  find optimal routes between two points in real time, \ie{} while
  you're driving.
\end{note}

\newpage

\begin{exercise}
  Use Dijkstra's algorithm to determine the shortest drive time from
  Seattle to Las Vegas. What is the total drive time? (The graph below
  is weighted by drive time in hours.)

  \bigskip

  \includestandalone{figures/seattle-to-las-vegas}

  \vspace*{\stretch{1}}
\end{exercise}

\newpage

\begin{exercise}
  Use Dijkstra's algorithm to determine the cheapest path from
  A to G.

  \bigskip

  \includestandalone{figures/dijkstra-example}

  \vspace*{\stretch{1}}
\end{exercise}

\newpage

\section{Hamiltonian Circuits and the Traveling Salesman \\ Problem}%
\label{sec:hamiltonian-circuits}

\begin{definition}
  A \textbf{Hamiltonian circuit} is a circuit that visits every vertex
  once with no repeats. Being a circuit, it must start and end at the
  same vertex. A \textbf{Hamiltonian path} also visits every vertex
  once with no repeats, but does not have to start and end at the same
  vertex.
\end{definition}



\subsection{Brute Force}%
\label{sub:brute-force}

\begin{algorithm}[Brute Force]
  \begin{enumerate}
  \item Create a list of all possible circuits and their costs.
  \item Choose the circuit with the least cost.
  \end{enumerate}
\end{algorithm}

\subsection{Nearest Neighbor Algorithm (NNA)}%
\label{sub:nearest-neighbor}

\begin{algorithm}[Nearest Neighbor Algorithm]
  \begin{enumerate}
  \item Select a starting point.
  \item Move to the nearest unvisited vertex (via the edge with
    smallest weight).
  \item Repeat until the circuit is complete.
  \end{enumerate}
\end{algorithm}

\begin{exercise}\label{ex:nna}
  Use the graph below to find a Hamiltonian path via NNA beginning at
  vertex B.

  \includestandalone{figures/nna}

  \vspace*{\stretch{1}}
\end{exercise}

\newpage

\subsection{Repeated Nearest Neighbor Algorithm (RNNA)}%
\label{sub:repeated-nearest-neighbor}

\begin{algorithm}[Repeated Nearest Neighbor Algorithm]
  \begin{enumerate}
  \item Do the Nearest Neighbor Algorithm starting at each vertex.
  \item Choose the circuit produced with minimal total weight.
  \end{enumerate}
\end{algorithm}

\begin{exercise}
  Use the graph to find a Hamiltonian path via RNNA. This is the same
  graph as used in Exercise~\ref{ex:nna}, so you may use your previous
  result where we did NNA starting at vertex B.


  \includestandalone{figures/nna}

  \vspace*{\stretch{1}}
\end{exercise}

\newpage

\subsection{Sorted Edges Algorithm (SEA)}%
\label{sub:sorted-edges}

\begin{algorithm}[Sorted Edges Algorithm]
  \begin{enumerate}
  \item Select the cheapest unused edge in the graph.
  \item Repeat step 1, adding the cheapest unused edge to the circuit,
    unless:
    \begin{enumerate}
    \item adding the edge would create a circuit that doesn't contain
      all vertices, or
    \item adding the edge would give a vertex degree 3.
    \end{enumerate}
  \item Repeat until a circuit containing all vertices is formed.
  \end{enumerate}
\end{algorithm}

\begin{exercise}\label{ex:sea}
  Use the graph to find a solution to the traveling salesman problem
  via the Sorted Edges Algorithm.

  \includestandalone{figures/sea}

  \vspace*{\stretch{1}}
\end{exercise}

\newpage

\section{Eulerian Circuits and the Chinese Postman Problem}%
\label{sec:eulerian-circuits}

The picture on the cover of these lecture notes is of the city of
K\:{o}nigsberg which is now called Kaliningrad. At the time that the
mathematician Leonhard Euler (pronounced like ``oiler'') lived there
there was a famous question that puzzled the residents. Is it possible
to start in one location and take a walk such that you cross each of
the seven bridges near the center of town just once and end up back
where you started?

\subsection{Fleury's Algorithm}%
\label{sub:Fleury's Algorithm}

\begin{algorithm}[Chinese Postman --- Fleury]
  \begin{enumerate}
  \item Start at any vertex if finding an Euler circuit. If finding an
    Euler path, start at one of the two vertices with odd degree.
  \item Choose any edge leaving your current vertex, provided deleting
    that edge will not separate the graph into two disconnected sets
    of edges.
  \item Add that edge to your circuit, and delete it from the graph.
  \item Continue until you're done.
  \end{enumerate}
\end{algorithm}

\section{Spanning Trees and Kruskal's Algorithm}%
\label{sec:spanning-trees}

\begin{definition}
  \begin{itemize}
  \item A \textbf{tree} is an undirected graph (edges are not arrows)
    in which any two vertices are connected by exactly one path.
  \item A \textbf{spanning tree} is an undirected subgraph of a graph
    \(G\) that is a tree which includes all the vertices of \(G\).
  \item A \textbf{minimum cost spanning tree} is the spanning tree
    with the smallest total edge weight.
  \end{itemize}
\end{definition}

\begin{algorithm}[Kruskal's Algorithm]
  \begin{enumerate}
  \item Select the cheapest unused edge in the graph.
  \item Repeat step 1, unless
    \begin{itemize}
    \item adding the edge would create a circuit, or
    \item until a spanning tree is formed.
    \end{itemize}
  \end{enumerate}
\end{algorithm}

\begin{note}
  Similar to Dijkstra's algorithm, Kruskal's algorithm is both
  \emph{optimal} (it finds the minimum cost spanning tree) and
  \emph{efficient} (it terminates in a reasonable amount of time).
\end{note}

\begin{exercise}
  A company requires reliable internet and phone connectivity between
  their five offices (named A, B, C, D and E for simplicity) in New
  York City. They decide to lease dedicated lines from the phone
  company. The phone company will charge for each link made. The costs
  in thousands of dollars per year are shown in the graph. Use
  Kruskal's algorithm to find a minimal cost spanning tree for the
  five offices.

  \includestandalone{figures/kruskal}
\end{exercise}






%%% Local Variables:
%%% mode: latex
%%% TeX-master: "Notes"
%%% End:
