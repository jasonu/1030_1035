\chapter{Voting Theory}%
\label{chap:voting-theory}

In many decision making situations, it is necessary to gather the
group consensus. This happens when a group of friends decides which
movie to watch, when a company decides which product design to
manufacture, and when a democratic country elects its leaders.

While the basic idea of voting is fairly universal, the method by
which those votes are used to determine a winner can vary. Amongst a
group of friends, you may decide upon a movie by voting for all the
movies you're willing to watch, with the winner being the one with the
greatest approval. A company might eliminate unpopular designs then
revote on the remaining. A country might look for the candidate with
the most votes.

In deciding upon a winner, there is always one main goal: to reflect
the preferences of the people in the most fair way possible. However,
we will see that there is more than one criteria for fairness. In
fact, there are several different criteria which can be used to
determine whether an election or decision process is fair or not.

In this chapter we will look at several methods for a group to make a
choice and analyze each method's strengths and weaknesses.

\section{Plurality Voting Method}%
\label{sec:plurality-method}

In the United States, most elections choose the winner based upon
which ever candidate receives the most votes. Notice that it is
\emph{not} necessary to receive a \emph{majority} (more than half) of
the votes, just more votes than any other candidate.

\begin{definition}
  The \textbf{plurality} voting method chooses the winner based on the
  candidate which receives more votes than any other candidate.
\end{definition}

\newpage

\begin{exercise}
  If Alex, Betty and Chuck are running for the office of club
  president, then what is the minimum number of votes that one could
  get to win the election under the plurality method if there are
  \begin{enumerate}
  \item 73 votes cast?
    \vspace*{\stretch{1}}
  \item 74 votes cast?
    \vspace*{\stretch{1}}
  \end{enumerate}
\end{exercise}

A two-party system often develops in a plurality voting system. In
this system, voters have a single vote, which they can cast for a
single candidate in their district, in which only one legislative seat
is available. In plurality voting (also referred to as
\textbf{first-past-the-post}), in which the winner of the seat is
determined purely by the candidate with the most votes, several
characteristics can serve to discourage the development of third
parties and reward the two major parties.

Because the first-past-the-post system gives only the (plurality)
winner in each district a seat, a party that consistently comes in
second or third in many or most districts will not gain any seats in
the legislature, even if it receives a substantial minority of the
vote.

Another challenge to a third party is both statistical and tactical.
Political scientist Maurice Duverger presents the example of an
election in which 100,000 moderate voters and 80,000 radical voters
are to vote for candidates for a single seat or office. If two
moderate parties ran candidates and one radical candidate ran (and
every voter voted), the radical candidate would tend to win unless one
of the moderate candidates gathered fewer than 20,000 votes.
Appreciating this risk, moderate voters would be inclined to vote for
the moderate candidate they deemed likely to gain more votes, with the
goal of defeating the radical candidate. To win, then, either the two
moderate parties must merge, or one moderate party must fail, as the
voters gravitate to the two strongest parties. Duverger called this
trend polarization.

\section{Instant Runoff Voting (Ranked Choice Voting)}%
\label{sec:irv}

What if we required the eventual winner to receive a \emph{majority}
in order to win? This may not be possible if there are more than two
candidates. A way around this problem is to do \textbf{runoff voting}
where the election is held in stages. After each vote, if a single
candidate has a majority of the votes then that candidate wins, but if
not, then the candidate with the least votes is eliminated and
everyone votes again. Clearly this is time consuming and inefficient.
One way around the inefficiencey of runoff voting is to modify the
ballot to allow voters to rank the candidates by order of preference.

\begin{definition}\label{def:preference-ballot-schedule}
  A \textbf{preference ballot} is a ballot in which the voter ranks
  the choices in order of preference.

  Once the election has been held we tally the votes and create a
  summary of voter preferences. We call this summary of the ballots a
  \textbf{preference schedule}.
\end{definition}

\begin{exercise}
  Below is an example of a preference schedule for candidates A,B,C and D.
  \begin{table}[h]
    \centering
    \begin{tabular}{ccccc}
      \toprule
      Num. Ballots & 8 & 17 & 4 & 11 \\
      \midrule
      1st & A & D & B & C \\
      2nd & B & A & A & B \\
      3rd & C & C & D & A \\
      4th & D & B & C & D \\
      \bottomrule
    \end{tabular}
    \caption{Example preference schedule}%
    \label{tab:example-preference-schedule}
  \end{table}
  Notice that there were a total of \( 8+17+4+11=40 \) votes cast.
  This means that for a candidate to get a \emph{majority} of the
  votes they must get at least 21 votes because 21 is the smallest
  whole number that is more than half of 40.

  \begin{enumerate}
  \item Who wins under the plurality method?

    \vspace*{\stretch{1}}

  \item Who is the winner under instant-runoff voting?

    \vspace*{\stretch{5}}

  \end{enumerate}
\end{exercise}

\newpage

\begin{exercise}
  Suppose there is an election with four candidates. Two are from the
  two dominant political parties: Democrat and Republican. The other
  two are from third parties: Libertarian and Socialist. A preference
  schedule is given below.
  \begin{center}
    \begin{tabular}{ccccc}
      \toprule
      Num. Ballots & 8 & 18 & 20 & 4 \\
      \midrule
      1st & L & R & D & S \\
      2nd & R & L & S & D \\
      3rd & D & S & L & R \\
      4th & S & D & R & L \\
      \bottomrule
    \end{tabular}
  \end{center}
  Answer the following questions using the data in the preference
  schedule above.
  \begin{enumerate}
  \item How many people voted?

    \vspace*{\stretch{1}}

  \item How many votes are required for a majority?

    \vspace*{\stretch{1}}

  \item What is the smallest possible number of votes a candidate
    could win with under the plurality method of choosing?

    \vspace*{\stretch{1}}


  \item Who wins under instant runoff voting?
    \vspace*{\stretch{5}}

  \end{enumerate}
\end{exercise}

\newpage

\section{Borda Count}%
\label{sec:borda-count}

Borda Count is another voting method, named for Jean-Charles de Borda,
who developed the system in 1770.

\begin{definition}
  In the \textbf{Borda Count} method, points are assigned to
  candidates based on their ranking; 1 point for last choice, 2 points
  for second-to-last choice, and so on. The point values for all
  ballots are totaled, and the candidate with the largest point total
  is the winner.
\end{definition}

\begin{exercise}\label{ex:borda-count}
  Suppose an election resulted in the following preference schedule.
  Which of the three candidates: A, B, or C would win under the Borda
  count method of choosing?
  \begin{center}
    \begin{tabular}{ccccc}
      \toprule
      Num. Ballots & 13 & 4 & 14 \\
      \midrule
      1st & B & A & C \\
      2nd & A & B & A \\
      3rd & C & C & B \\
      \bottomrule
    \end{tabular}
  \end{center}

  \vspace*{\stretch{1}}
\end{exercise}

\begin{exercise}\label{ex:irv}
  Which candidate would win under instant runoff voting?

  \vspace*{\stretch{1}}
\end{exercise}

\newpage

\section{Copeland's Method (Pairwise Comparison)}%
\label{sec:copelands-method}

Sometimes a candidate can be preferred by voters when compared to
every other candidate, but still not win the election. This seems
unfair to most people.

\begin{definition}
  If one candidate is preferable to all other candidates in
  head-to-head matchups, then that candidate is called the
  \textbf{Condorcet candidate}.
\end{definition}

\begin{exercise}\label{ex:copeland}
  Use the preference schedule from exercise~\ref{ex:borda-count} to
  determine if there was a Condorcet candidate in that election. The
  table is reproduced below.
  \begin{center}
    \begin{tabular}{ccccc}
      \toprule
      Num. Ballots & 13 & 4 & 14 \\
      \midrule
      1st & B & A & C \\
      2nd & A & B & A \\
      3rd & C & C & B \\
      \bottomrule
    \end{tabular}
  \end{center}

  \vspace*{\stretch{1}}
\end{exercise}



\section{Fairness Criteria}%
\label{sec:fairness criteria}

\begin{definition}
  A \textbf{fairness criterion} is a conditional statement (if---then)
  about a selection or voting method that seems like it should be
  satisfied in order for the method to be fair.
\end{definition}

\begin{description}
\item[Plurality Criterion] \hfill \\
  If there is a candidate that gets more votes than any other
  candidate, then that candidate should win.
\item[Majority Criterion] \hfill \\
  If there is a candidate that has a majority (more than half) of
  first-place votes, then that candidate should win.
\item[Condorcet Criterion] \hfill \\
  If there is a candidate that wins every head-to-head comparison,
  then that candidate should win.
\item[Monotonicity Criterion] \hfill \\
  If a candidate wins, and only changes that favor the winner are made
  to the preference ballots, then that candidate should still win.
\item[Independence of Irrelevant Alternatives (IIA) Criterion] \hfill \\
  If a candidate wins, and only losing candidates are removed from the
  preference ballots, then that candidate should still win.
\end{description}


\section{Arrow's Impossibility Theorem}%
\label{sec:arrows-impossibility-thm}





%%% Local Variables:
%%% mode: latex
%%% TeX-master: "Notes"
%%% End:
