\chapter{Voting Theory}%
\label{chap:voting-theory}

In many decision making situations, it is necessary to gather the
group consensus. While the basic idea of voting is fairly universal,
many methods for determining a winner have been created.

In deciding upon a winner, there is always one main goal: to reflect
the preferences of the people in the most fair way possible. However,
as we will see, there is no single criterion for determining whether a
voting method is fair. In fact, there are several different criteria
which can be used to determine whether a voting method is fair or not.

In this chapter we will look at several voting methods and analyze
each method's strengths and weaknesses.

\section{Plurality Voting Method}%
\label{sec:plurality-method}

In the United States and in many other countries throughout the world,
most elections are decided by which ever candidate receives the most
votes. Notice that it is \emph{not} necessary to receive a
\emph{majority} (more than half) of the votes, just more votes than
any other candidate.

\begin{definition}
  The \textbf{plurality} voting method chooses the winner based on the
  candidate which receives more votes than any other candidate. This
  method is also known as \textbf{first-past-the-post} in an analogy
  to horse racing.
\end{definition}

\begin{exercise}
  If Alex, Betty and Chuck are running for the office of club
  president, then what is the minimum number of votes that one could
  get to win the election under the plurality method if there are
  \begin{enumerate}
  \item 73 votes cast?
    \vspace*{\stretch{1}}
  \item 74 votes cast?
    \vspace*{\stretch{1}}
  \end{enumerate}
\end{exercise}

\newpage

When there are only two candidates or choices, plurality voting works
extremely well, but as soon as there are three or more candidates or
choices, problems can arise.

The first problem is that under plurality voting, the winner may not
receive a majority of the votes. This seems unfair to many voters. The
second problem is that plurality voting tends to discourage the
development of third parties and reward the two major parties. This is
because plurality voting gives only the winner in each district a
seat, a party that consistently comes in second or third in many or
most districts will not gain any seats in the legislature, even if it
receives a substantial minority of the vote.

\begin{definition}
  \textbf{Tactical voting}, or \textbf{insincere voting} occurs in
  elections with more than two candidates, when a voter supports
  another candidate more strongly than their sincere preference in
  order to prevent an undesirable outcome.
\end{definition}

Because of tactical or insincere voting, voters sometimes feel like
they must choose between the lesser of two evils. This can lead to a
general lack of enthusiasm for voting.

Lack of enthusiasm is not the only problem that plurality voting tends
to create. Plurality voting sometimes leads to polarization of the
electorate into supporting just two parties.

Consider an election in which 100,000 moderate voters and 80,000
radical voters are to vote for candidates for a single seat or office.
If two moderate parties ran candidates and one radical candidate ran
(and every voter voted), the radical candidate would tend to win
unless one of the moderate candidates gathered fewer than 20,000
votes. Appreciating this risk, moderate voters would be inclined to
vote for the moderate candidate they deemed likely to gain more votes,
with the goal of defeating the radical candidate. To win, then, either
the two moderate parties must merge, or one moderate party must fail,
as the voters gravitate to the two strongest parties. If this scenario
happens enough times it can push voters to only support the two
strongest parties and third parties are often (but not always) shut
out of the electoral process.

Plurality voting is a rather blunt tool for determining the will of
the people when there are several candidates. This is because each
voter only gives one piece of information at the polls, their favorite
candidate, and due to insincere voting, that information may not even
be accurate. What if we gathered more information from each voter
about their preferences? Perhaps the extra information could be used
to generate fairer outcomes. This is the general idea behind
\textbf{ranked choice voting} methods. Many of the voting methods that
have been created use this idea to try to make elections fairer.

\newpage

\section{Instant Runoff Voting}%
\label{sec:irv}

What if we required the eventual winner to receive a \emph{majority}
in order to win? This may not be possible if there are more than two
candidates. A way around this problem is to do \textbf{runoff voting}
where the election is held in stages. After each vote, if a single
candidate has a majority of the votes then that candidate wins, but if
not, then the candidate with the least votes is eliminated and
everyone votes again.

Runoff voting is time consuming and inefficient. One way around the
inefficiency of runoff voting is to modify the ballot to allow voters
to rank the candidates by order of preference.

\begin{definition}\label{def:preference-ballot-schedule}
  A \textbf{preference ballot} is a ballot in which the voter ranks
  the choices in order of preference. Once the election has been held
  we tally the votes and create a tabular summary of voter
  preferences. We call this tabular summary of the ballots a
  \textbf{preference schedule}.
\end{definition}

\begin{exercise}
  Below is an example of a preference schedule for candidates A,B,C
  and D. Notice that there were a total of \( 8+17+4+11=40 \) votes
  cast. This means that for a candidate to get a \emph{majority} of
  the votes they must get at least 21 votes because 21 is the smallest
  whole number that is more than half of 40.
  \begin{table}[h]
    \centering
    \begin{tabular}{ccccc}
      \toprule
      Num. Ballots & 8 & 17 & 4 & 11 \\
      \midrule
      1st & A & D & B & C \\
      2nd & B & A & A & B \\
      3rd & C & C & D & A \\
      4th & D & B & C & D \\
      \bottomrule
    \end{tabular}
    \caption{Example preference schedule}%
    \label{tab:example-preference-schedule}
  \end{table}

  \begin{enumerate}
  \item Who wins under the plurality method?

    \vspace*{\stretch{1}}

  \item Who is the winner under instant-runoff voting?

    \vspace*{\stretch{5}}

  \end{enumerate}
\end{exercise}

\newpage

\begin{exercise}
  Suppose there is an election with four candidates. Two are from the
  two dominant political parties: Democrat and Republican. The other
  two are from third parties: Libertarian and Socialist. A preference
  schedule is given below.
  \begin{center}
    \begin{tabular}{ccccc}
      \toprule
      Num. Ballots & 8 & 18 & 20 & 4 \\
      \midrule
      1st & L & R & D & S \\
      2nd & R & L & S & D \\
      3rd & D & S & L & R \\
      4th & S & D & R & L \\
      \bottomrule
    \end{tabular}
  \end{center}
  Answer the following questions using the data in the preference
  schedule above.
  \begin{enumerate}
  \item How many people voted?

    \vspace*{\stretch{1}}

  \item How many votes are required for a majority?

    \vspace*{\stretch{1}}

  \item What is the smallest possible number of votes a candidate
    could win with under the plurality method of choosing?

    \vspace*{\stretch{1}}


  \item Who wins under instant runoff voting?
    \vspace*{\stretch{5}}

  \end{enumerate}
\end{exercise}

\newpage

\section{Borda Count}%
\label{sec:borda-count}

Borda Count is another voting method that uses a preference ballot. It
is named after Jean-Charles de Borda, who developed the system in
1770.

\begin{definition}
  In the \textbf{Borda Count} method, points are assigned to
  candidates based on their ranking; 1 point for last choice, 2 points
  for second-to-last choice, and so on. The point values for all
  ballots are totaled, and the candidate with the largest point total
  is the winner.
\end{definition}

\begin{exercise}\label{ex:borda-count}
  Suppose an election resulted in the following preference schedule.
  Which of the three candidates: A, B, or C would win under the Borda
  count method of choosing?
  \begin{center}
    \begin{tabular}{ccccc}
      \toprule
      Num. Ballots & 13 & 4 & 14 \\
      \midrule
      1st & B & A & C \\
      2nd & A & B & A \\
      3rd & C & C & B \\
      \bottomrule
    \end{tabular}
  \end{center}

  \vspace*{\stretch{1}}
\end{exercise}

\begin{exercise}\label{ex:irv}
  Which candidate would win under instant runoff voting?

  \vspace*{\stretch{1}}
\end{exercise}

\newpage

\section{Copeland's Method (Pairwise Comparison)}%
\label{sec:copelands-method}

Sometimes a candidate can be preferred by voters when compared to
every other candidate, but still not win the election. This seems
unfair to most people.

\begin{definition}
  If one candidate is preferable to all other candidates in
  head-to-head match ups, then that candidate is called the
  \textbf{Condorcet candidate}.
\end{definition}

\begin{note}
  Not every election has a Condorcet candidate.
\end{note}

\begin{exercise}\label{ex:copeland}
  The preference schedule from exercise~\ref{ex:borda-count} is
  reproduced below.
  \begin{center}
    \begin{tabular}{ccccc}
      \toprule
      Num. Ballots & 13 & 4 & 14 \\
      \midrule
      1st & B & A & C \\
      2nd & A & B & A \\
      3rd & C & C & B \\
      \bottomrule
    \end{tabular}
  \end{center}

  \begin{enumerate}
  \item Determine if this election had a Condorcet candidate.

    \vspace*{\stretch{1}}

  \item In exercise~\ref{ex:irv} we determined the winner of this
    election under instant-runoff-voting. Is that winner and the
    Condorcet candidate the same person? Do you think this is fair?

    \vspace*{\stretch{1}}
  \end{enumerate}
\end{exercise}

\newpage

\begin{definition}
  In \textbf{Copeland's method} we award points to each candidate
  based upon how that candidate does in pairwise comparisons with
  every other candidate. The candidate with the most points wins.
  \begin{center}
    \begin{tabular}[h]{l}
      \toprule
      1 point for pairwise win \\ \\
      \(\tfrac{1}{2}\) point for pairwise tie \\ \\
      0 points for pairwise loss \\
      \bottomrule
    \end{tabular}
  \end{center}
\end{definition}

\begin{exercise}
  Use Copeland's method to determine the winner of the following
  preference schedule.

  \begin{center}
    \begin{tabular}[h]{ccccc}
      \toprule
      Num. Ballots & 13 & 15 & 9 & 11 \\
      \midrule
      1st & C & A & D & B \\
      2nd & D & B & B & C \\
      3rd & A & C & A & D \\
      4th & B & D & C & A \\
      \bottomrule
    \end{tabular}
  \end{center}

  \vspace*{\stretch{1}}
\end{exercise}

\newpage

\section{Fairness Criteria}%
\label{sec:fairness criteria}

\begin{definition}
  A \textbf{fairness criterion} is a conditional statement (if---then)
  about a selection or voting method that seems like it should be
  satisfied in order for the method to be fair.
\end{definition}

\begin{description}
% \item[Plurality Criterion] \hfill \\
%   If there is a candidate that gets more votes than any other
%   candidate, then that candidate should win.
\item[Majority Criterion] \hfill \\
  If there is a candidate that has a majority (more than half) of
  first-place votes, then that candidate should win.
\item[Condorcet Criterion] \hfill \\
  If there is a candidate that wins every head-to-head comparison,
  then that candidate should win.
\item[Monotonicity Criterion] \hfill \\
  If a candidate wins, and only changes that favor the winner are made
  to the preference ballots, then that candidate should still win.
\item[Independence of Irrelevant Alternatives (IIA) Criterion] \hfill \\
  If a non-winning candidate is removed from the election, then that
  should not change the winner.
\end{description}

\begin{table}[h!]
  \centering
  \begin{tabular}{ccccc}
    \toprule
    & \multicolumn{4}{c}{Fairness Criteria} \\
    Voting Method & Majority & Condorcet & Monotonicity & IIA \\
    \toprule
    Plurality & \cmark{} & \xmark{} & \cmark{} & \xmark{} \\
    \midrule
    Instant Runoff & \cmark{} & \xmark{} & \xmark{} & \xmark{} \\
    \midrule
    Borda Count & \xmark{} & \xmark{} & \cmark{} & \xmark{} \\
    \midrule
    Copeland & \cmark{} & \cmark{} & \cmark{} & \xmark{} \\
    \bottomrule
  \end{tabular}
  \caption{Voting Methods and Fairness Criteria \\ \cmark{} = always
    satisfied \\ \xmark{} = may violate}%
  \label{tab:voting-methods-fairness-criteria}
\end{table}

\newpage

\begin{example}[IRV Can Violate the Monotonicity Criterion]
  Consider the preference schedule below.
  \begin{center}
    \begin{tabular}[h]{ccccccc}
      \toprule
      Num. Ballots & 37 & 22 & 12 & 29 \\
      \midrule
      1st & Adams & Brown & Brown & Carter \\
      2nd & Brown & Carter & Adams & Adams \\
      3rd & Carter & Adams & Carter & Brown \\
      \bottomrule
    \end{tabular}
  \end{center}
  Under instant runoff voting, Carter would be eliminated in the first
  round, and Adams would be the winner with 66 votes to 34 votes for
  Brown.

  Now suppose that the results were announced, but election officials
  accidentally destroyed the ballots before they could be certified,
  and the votes had to be recast. Wanting to ``jump on the bandwagon'',
  10 of the voters who originally voted in the order Brown, Adams,
  Carter change their vote to favor the presumed winner, changing
  those votes to Adams, Brown, Carter.
  \begin{center}
    \begin{tabular}[h]{ccccccc}
      \toprule
      Num. Ballots & 47 & 22 & 2 & 29 \\
      \midrule
      1st & Adams & Brown & Brown & Carter \\
      2nd & Brown & Carter & Adams & Adams \\
      3rd & Carter & Adams & Carter & Brown \\
      \bottomrule
    \end{tabular}
  \end{center}
  In this re-vote, Brown will be eliminated in the first round, having
  the fewest first-place votes. After transferring votes, we find that
  Carter will win this election with 51 votes to Adams' 49 votes! Even
  though the only vote changes made favored Adams, the change ended up
  costing Adams the election. This doesn't seem fair. If voters change
  their votes to increase the preference for a candidate, it should
  not harm that candidate's chances of winning.
\end{example}

\newpage

\begin{example}[Copeland Can Violate the IIA Criterion]
  A committee is trying to award a scholarship to one of four
  students, Anna (A), Brian (B), Carlos (C), and Dimitry (D). The
  preference schedule is shown below:
  \begin{center}
    \begin{tabular}[h]{ccccccc}
      \toprule
      Num. Ballots & 5 & 5 & 6 & 4 \\
      \midrule
      1st & D & A & C & B \\
      2nd & A & C & B & D \\
      3rd & C & B & D & A \\
      4th & B & D & A & C \\
      \bottomrule
    \end{tabular}
  \end{center}
  Computation of Copeland point totals is as follows:
  \begin{center}
    \begin{tabular}[h!]{cccccc}
      \toprule
      Pair & Tallies & A & B & C & D \\
      \toprule
      A vs. B & 10 vs. 10 & 0.5 & 0.5 & & \\
      A vs. C & 14 vs. 6  &   1 &     &   & \\
      A vs. D & 5 vs. 15  &     &     &   & 1 \\
      B vs. C & 4 vs. 16  &     &     & 1 & \\
      B vs. D & 15 vs. 5  &     &   1 &   & \\
      C vs. D & 11 vs. 9  &     &     & 1 & \\
      \midrule

      \multicolumn{2}{c}{Totals} & 1.5 & 1.5 & 2 & 1 \\
      \bottomrule
    \end{tabular}
  \end{center}

  So Carlos is awarded the scholarship. However, the committee then
  discovers that Dimitry was not eligible for the scholarship (he failed
  his last math class). Even though this seems like it shouldn't affect
  the outcome, the committee decides to recount the vote, removing
  Dimitry from consideration. This reduces the preference schedule to:
  \begin{center}
    \begin{tabular}[h]{ccccccc}
      \toprule
      Num. Ballots & 5 & 5 & 6 & 4 \\
      \midrule
      1st & A & A & C & B \\
      2nd & C & C & B & A \\
      3rd & B & B & A & C \\
      \bottomrule
    \end{tabular}
  \end{center}
  Re-computation of Copeland point totals is as follows:
  \begin{center}
    \begin{tabular}[h!]{ccccc}
      \toprule
      Pair & Tallies & A & B & C \\
      \toprule
      A vs. B & 10 vs. 10 & 0.5 & 0.5 & \\
      A vs. C & 14 vs. 6  &   1 &     & \\
      B vs. C & 4 vs. 16  &     &     & 1 \\
      \midrule
      \multicolumn{2}{c}{Totals} & 1.5 & 0.5 & 1 \\
      \bottomrule
    \end{tabular}
  \end{center}
  Suddenly Anna is the winner!
\end{example}

\newpage

\section{Arrow's Impossibility Theorem}%
\label{sec:arrows-impossibility-thm}

Economist Kenneth Arrow was able to prove in 1949 that when an
election has three or more candidates, no ranked choice voting method
can simultaneously satisfy all fairness criteria. He won the 1972
Nobel Prize in Economics for this work.

\begin{theorem}[Arrow's Impossibility Theorem]
No voting method can satisfy all fairness criteria.
\end{theorem}

\subsection{Approval Voting}%
\label{sub:approval-voting}

\begin{definition}
\textbf{Approval voting} is where you mark all candidates you approve of
without stating preference, \ie{} without ranking the candidates.
\end{definition}

\begin{exercise}
  Determine the winner under approval voting.
  \begin{center}
    \begin{tabular}[h]{ccccccc}
      \toprule
      &\multicolumn{6}{c}{Number of Ballots}\\
      Choice&10&21&17&13&18&16\\
      \toprule
      A&&&&\cmark&\cmark&\cmark\\
      \midrule
      B&\cmark&\cmark&\cmark&&&\\
      \midrule
      C&&\cmark&&\cmark&&\cmark\\
      \midrule
      D&\cmark&\cmark&&&\cmark&\\
      \bottomrule
    \end{tabular}
  \end{center}

  \vspace*{\stretch{1}}
\end{exercise}

\newpage

\begin{table}[h!]
  \centering
  \begin{tabular}{lcccccccccc}
    \toprule
    Num. Ballots &Bob&Ann&Marv&Alice&Eve&Omar&Lupe&Dave&Tish&Jim\\
    \toprule
    Titanic &&\cmark&\cmark&&&\cmark&&\cmark&\cmark&\cmark\\
    Scream &\cmark&&\cmark&\cmark&&\cmark&\cmark&&\cmark&\\
    The Matrix &\cmark&\cmark&\cmark&\cmark&\cmark&&\cmark&&&\cmark\\
    \bottomrule
  \end{tabular}
  \caption[Movie choice]{Using approval voting to choose a movie to watch.}%
  \label{tab:movie-choice}
\end{table}


\begin{exercise}
  Use Table~\ref{tab:movie-choice} and approval voting to determine
  the movie to watch.

  \vspace*{\stretch{1}}
\end{exercise}

This is an easier method of voting because it does not force the voter
to rank choices. The negatives to approval voting are twofold. First,
it can easily violate the \emph{majority} fairness criterion. Second,
approval voting is susceptible to strategic insincere voting, in
whicha voter does not voter their true preference to try to increase
the chances of their choice winning. For example, suppose Bob and
Alice would much rather watch ``Scream''. Even though they may be okay
with watching ``The Matrix'', if they remove it from their approval
list, its chances of winning decrease and thus their chance to watch
``Scream'' increases.

\newpage



%%% Local Variables:
%%% mode: latex
%%% TeX-master: "Notes"
%%% End:
