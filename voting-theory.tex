\chapter{Voting Theory}%
\label{chap:voting-theory}

\section{Plurality Method}%
\label{sec:plurality-method}

\begin{exercise}
  If Alex, Betty and Chuck are running for the office of club
  president, then what is the minimum number of votes that one could
  get to win the election under the plurality method if there are
  \begin{enumerate}
  \item 73 votes cast?
    \vspace*{\stretch{1}}
  \item 74 votes cast?
    \vspace*{\stretch{1}}
  \end{enumerate}
\end{exercise}

\section{Instant Runoff Voting (Ranked Choice Voting)}%
\label{sec:irv}

\subsection{Preference Schedules}%
\label{sub:preference-schedules}

\section{Borda Count}%
\label{sec:borda-count}


\section{Copeland's Method (Pairwise Comparison)}%
\label{sec:copelands-method}


\section{Fairness Criteria}%
\label{sec:fairness criteria}

\begin{definition}
  A \textbf{fairness criterion} is a conditional statement (if---then)
  about a selection or voting method that seems like it should be
  satisfied in order for the method to be fair.
\end{definition}

\begin{description}
\item[Plurality Criterion] \hfill \\
  If there is a candidate that gets more votes than any other
  candidate, then that candidate should win.
\item[Majority Criterion] \hfill \\
  If there is a candidate that has a majority (more than half) of
  first-place votes, then that candidate should win.
\item[Condorcet Criterion] \hfill \\
  If there is a candidate that wins every head-to-head comparison,
  then that candidate should win.
\item[Monotonicity Criterion] \hfill \\
  If a candidate wins, and only changes that favor the winner are made
  to the preference ballots, then that candidate should still win.
\item[Independence of Irrelevant Alternatives (IIA) Criterion] \hfill \\
  If a candidate wins, and only losing candidates are removed from the
  preference ballots, then that candidate should still win.
\end{description}


\section{Arrow's Impossibility Theorem}%
\label{sec:arrows-impossibility-thm}





%%% Local Variables:
%%% mode: latex
%%% TeX-master: "Notes"
%%% End:
