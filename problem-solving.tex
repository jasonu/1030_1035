\chapter{Problem Solving}%
\label{chap:problem-solving}

\section{Exploiting Fractions for Fun and Profit}
\label{sec:proportions-rates}

\subsection{Absolute and Relative Change}
\label{sub:absolute-relative-change}

In this section we will see how to use the two arithemtic operations
of subtraction and division to measure how something changes over
time.
\begin{definition}
  \textbf{Absolute change} is the difference:
  \[
    \text{absolute change} = \text{new} - \text{old}
  \]
\end{definition}

Absolute change can be useful for comparing things that are similar.
For example you could compare how many inches two fourteen year old
boys grew in one year. But suppose the things you wish to compare are
related but on very different size scales.

\begin{example}[Stock Investments Part I]
  Suppose you invest \$50,000 in Acme corporation stock and after one
  year the stock is now worth \$52,000. This is an absolute change of
  \$2,000. Now suppose you also invest \$500 in Hooli corporation and
  after one year the stock is worth \$550. Which investment was better?
\end{example}
\begin{solution}
  \begin{align*}
    \text{absolute change in Acme} &= \$52,000 - \$50,000 = \$2,000 \\
    \text{absolute change in Hooli} &= \$550 - \$500 = \$50
  \end{align*}
  In terms of absolute change the yield from the Acme investment is
  larger, so it is better right? Well what if we had switched our
  investment strategy and swapped the initial stock purchase for each
  company? To answer this question we need to see how big the absolute
  change in value was relative to the initial outlay of money. We need
  relative change.
\end{solution}

\begin{definition}
  \textbf{Relative change} is the ratio of absolute change to the original (old) value.
  \[
    \text{relative change} = \frac{\text{new} - \text{old}}{\text{old}}
  \]
\end{definition}

\begin{example}[Stock Investments Part II]
What is the relative change in value of each stock purchase?
\end{example}
\begin{solution}
  \begin{align*}
    \text{relative change in Acme}
    &= \frac{\$52,000 - \$50,000}{\$50,000} = \frac{\$2,000}{\$50,000} = 0.04 \\
    \text{relative change in Hooli}
    &= \frac{\$550 - \$500}{\$500} = \frac{\$50}{\$500} = 0.10
  \end{align*}
  Hooli's relative change was higher than Acme's relative change, and
  thus the Hooli stock was the better investment.
\end{solution}

\begin{note}
  Notice that absolute change has units attached to the value. In the
  above examples the units were dollars (\$). But when you compute
  relative change, the numerator and denominator will have the same
  units and thus the units will cancel and you are left with a pure
  (unitless) number.
\end{note}

\subsection{Percentages}
\label{sub:percentages}

\begin{definition}
  A \textbf{percent} or \textbf{percentage} is a fraction, where the
  denominator is 100, but for ease of writing we drop the denominator
  and replace it with the percent symbol \(\%\). For example,
  \[
    \frac{3}{4} = 0.75 = \frac{75}{100} = 75\%
  \]
  Percentages can be negative and even larger than 100, for example
  \[
    250\% = \frac{250}{100} = 2.5
  \]
\end{definition}

\begin{exercise}
  Express the following values as percentages.
  \begin{enumerate}
  \item \(\displaystyle\frac{1}{5}\)
    \vspace*{\stretch{1}}
  \item \(\displaystyle 0.004\)
    \vspace*{\stretch{1}}
  \item \(\displaystyle 3.11\)
    \vspace*{\stretch{1}}
  \end{enumerate}
\end{exercise}

It is often convenient to express relative change as a percent
\begin{example}[Stock Investments Part III]
  What was the percent change in value of each stock investment?
  \begin{center}
    \begin{tabular}{rccc}
      \toprule
      Company & Decimal & Fraction & Percent \\
      \midrule
      Acme & .04 & \(\frac{4}{100}\) & 4\% \\ \\
      Hooli & .10 & \(\frac{10}{100}\) & 10\% \\
      \bottomrule
    \end{tabular}
\end{center}
\end{example}

\newpage

\begin{exercise}
  Your truck was worth \$28,000 in 2019 and is now worth \$23,500 in
  2021. What is the percent change in value of your truck?

  \vspace*{\stretch{1}}
\end{exercise}


\begin{exercise}
  Your \$290,000 home increased in value by 5\%. How much is it worth now?

  \vspace*{\stretch{1}}
\end{exercise}

\begin{exercise}
  A store has a clearance rack where every item is marked down 20\%
  off the original price. They have a sale advertising an additional
  20\% off everything in the store. What percentage of the original
  price do you end up paying?

  \vspace*{\stretch{1}}
\end{exercise}

\newpage

\subsection{Rates}
\label{sub:rates}

\begin{definition}
  A \text{rate} is a ratio of two quantities. Rates are also known as
  fractions.
\end{definition}

\begin{exercise}
  If your car can travel \US{284}{\mile} per fillup and your gas tank
  holds \US{9.5}{\gallon}, then at what rate does your car consume
  gasoline, or more commonly, what is its mpg?

  \vspace*{\stretch{1}}
\end{exercise}

\subsection{Proportionality}
\label{sub:proportionality}

\begin{definition}
  A \text{proportion} is a ratio of two quantities. Thus it is a fraction.
\end{definition}

\begin{exercise}
  A map's legend indicates that \US{\tfrac{1}{2}}{\inch} equals
  \US{4}{\mile}. How many miles apart in real life are two points on
  the map that are separated by \US{3\tfrac{1}{2}}{\inch}?

  \vspace*{\stretch{1}}
\end{exercise}

\begin{exercise}
  A cookie recipe calls for 2 cups of chocolate chips per batch of 24
  cookies. If you wish to make 216 cookies?

  \vspace*{\stretch{1}}
\end{exercise}


\section{Exploiting Units for Fun and Profit}
\label{sec:exploiting-units}

\subsection{Unit Conversions}
\label{sub:unit-conversions}

\subsection{Geometry: Area and Volume Formulas}


\subsection{Linear, Areal and Volumetric Densities}
\label{sub:densities}





%%% Local Variables:
%%% mode: latex
%%% TeX-master: "Notes"
%%% End:
