\chapter{Growth Models}%
\label{chap:growth-models}

\begin{definition}
  A \textbf{recursive} function is a function defined in terms of its
  previous value(s).
\end{definition}

\begin{note}
  Recursive functions are \emph{easy to define}, but \emph{tedious to
    evaluate} because you often have to compute several previous
  values. For this reason, they are often used in computer programs,
  but not when doing computation by hand.
\end{note}

\begin{example}[Fibonacci Sequence]
  The famous Fibonacci function, \(F\), is defined \emph{recursively}
  as follows:
  \[
    F(x) = F(x-2)+F(x-1).
  \]

  Defining \(F(0)=0\) and \(F(1)=1\) yields the following sequence of
  outputs:
  \begin{align*}
    F(0) &= 0 \\
    F(1) &= 1 \\
    F(2) &= F(0) + F(1) = 0+1 = 1 \\
    F(3) &= F(1) + F(2) = 1+1 = 2 \\
    F(4) &= F(2) + F(3) = 1+2 = 3 \\
    F(5) &= F(3) + F(4) = 2+3 = 5 \\
    F(6) &= F(4) + F(5) = 3+5 = 8 \\
    F(7) &= F(5) + F(6) = 5+8 = 13 \\
     & \: \: \: \vdots
  \end{align*}
  Each number in the sequence, is the sum of the previous two numbers:
  \(0, 1, 1, 2, 3, 5, 8, 13, 21, 34\ldots\)
\end{example}

\section{Linear Growth}%
\label{sec:linear-growth}

\subsection{Recursive Linear Growth Model}%
\label{sec:recursive-linear}

Let \(P=\) population, \(x=\) time in years, then we predict next
years population, \(P(x+1)\), will be this year's population,
\(P(x)\), plus a fixed amount, \(d\).

\begin{equation}
  \boxed{P(x+1) = P(x) + d}\label{eq:recursive-linear-model}
\end{equation}

Thus \textbf{linear growth} (or decay) is characterized by the
\emph{difference} in population from two \emph{consecutive} years
equaling the constant: \(d\).
\[
  \boxed{P(x+1) - P(x) = d}
\]

\subsection{Explicit Linear Growth Model}%
\label{sec:explicit-linear}

\begin{figure}[b]
  \centering
  \includestandalone[scale=0.8]{figures/lin-growth-varying-m}
  \includestandalone[scale=0.8]{figures/lin-growth-varying-b}
  \caption{Graphs of exponential growth, \(y=mx+b\)}%
  \label{fig:linear-growth}
\end{figure}

Our goal in this section is to show that the recursive definition of
linear growth is equivalent to familiar equational model of lines:
\(y=mx+b\) where \(m\) is the slope and \(b\) gives the
\(y\)-intercept of the graph, see Figure~\ref{fig:linear-growth}.

Let \(P_0\equiv P(0)\), and recall the recursive definition:
\(P(x+1) = P(x) + d \):
\begin{align*}
  P(1) &= P(0) + d = P_{0} + d & P(1) &= P_{0} + d\\
  P(2) &= P(1) + d = (P_{0} + d) + d & P(2) &= P_{0} + 2d \\
  P(3) &= P(2) + d = (P_{0} + 2d) + d & P(3) &= P_{0} + 3d \\
  P(4) &= P(3) + d = (P_{0} + 3d) + d & P(4) &= P_{0} + 4d \\
       & \: \: \vdots && \\
  P(x) &= P_{0} + xd &&\\
  \intertext{Or equivalently,}
  \Aboxed{P(x) &= dx + P_0} &&
\end{align*}
where \(d\) is the slope or growth rate, and \(P_0\) is the
\(y\)--intercept, or initial population.

\newpage

\section{Exponential Growth}%
\label{sec:exponential-growth}

\subsection{Recursive Exponential Growth Model}%
\label{sec:recursive-exponential}

Again, let \(P=\) population, \(x=\) time measured in years, then we
predict the population next year, \(P(x+1)\), to be the population
this year, \(P(x)\), plus a fixed percentage, \(r\), of this year's
population:
\begin{equation}
  P(x+1) = P(x) + rP(x) \qquad \Rightarrow \qquad
  \boxed{P(x+1) = P(x)(1+r)}
\end{equation}
Thus \textbf{exponential growth} (or decay) is characterized by the
\emph{ratio} of populations from \emph{consecutive} years, equaling
the constant: \(1+r\).
\[
  \boxed{\frac{P(x+1)}{P(x)} = 1+r}
\]

\subsection{Explicit Exponential Growth Model}%
\label{sub:explicit-exponential}

Let \(P_{0}\equiv P(0)\), and recall the recursive definition: \(P(x+1) = P(x)(1+r) \)
\begin{align*}
  P(1) &= P_{0}(1+r) & P(1) &= P_{0}{(1+r)}^{1}\\
  P(2) &= P(1)(1+r) = P_{0}(1+r)(1+r) & P(2) &= P_{0}{(1+r)}^2 \\
  P(3) &= P(2)(1+r) = P_{0}{(1+r)}^2(1+r) & P(3) &= P_{0}{(1+r)}^3 \\
  P(4) &= P(3)(1+r) = P_{0}{(1+r)}^3(1+r) & P(4) &= P_{0}{(1+r)}^4 \\
       & \: \: \vdots &&\\
  \Aboxed{P(x) &= P_0{(1+r)}^x} &&
\end{align*}
where \(r\) is the growth rate, and \(P_0\) is the \(y\)--intercept,
or initial population.
\begin{figure}[b]
  \centering
  \includestandalone[scale=0.8]{figures/exp-growth-varying-b}
  \includestandalone[scale=0.8]{figures/exp-growth-varying-a}
  \caption{Graphs of exponential growth, \(y=ab^{x}\)}%
  \label{fig:exponential-growth}
\end{figure}

\newpage

\subsection{Summary of Growth Models}%
\label{sub:summary-growth-models}

\begin{center}
  {\renewcommand{\arraystretch}{2}
    \begin{tabular}[h!]{rcc}
      \toprule
      & Linear & Exponential \\
      \toprule
      Recursive Form & \(P(x+1) = P(x)+d\) & \(P(x+1) = P(x)(1+r)\) \\
      \midrule
      Constant & \(P(x+1) - P(x) = d\)  & \(\dfrac{P(x+1)}{P(x)} = 1+r\) \\
      \midrule
      Explicit Form & \(P(x) = dx + P_0\) & \(P(x) = P_0{(1+r)}^x \) \\
      \midrule
      Alternate Form & \(y = mx + b\) & \(y = a b^{x}\) \\
      \bottomrule
    \end{tabular}
  }
\end{center}

\begin{exercise}
  Suppose your company is building a solar plant. The plant had 112
  solar panels in 2007 and had 884 in 2018. Assuming the growth is
  \emph{linear}, find an explicit formula for the growth of solar
  panels.

  \vspace*{\stretch{1}}
\end{exercise}

\newpage

\begin{exercise}
  Suppose you bought a house for \$193,000 and it is worth \$299,000
  in 2018.

  \begin{enumerate}
  \item Find the rate of growth and common ratio of you bought the
    house in 1998.

    \vspace*{\stretch{1}}
  \item Find the rate of growth and common ratio of you bought the
    house in 2008.

    \vspace*{\stretch{1}}
  \end{enumerate}
\end{exercise}

\newpage

\section{Logarithms}%
\label{sec:logarithms}

A \textbf{logarithm} (with base \(b\)) is the inverse of
exponentiation (with base \(b\)), symbolically:
\[
  y = {b}^{x} \quad \Leftrightarrow \quad x = \log_{b}(y).
\]
\begin{example}
  The following two equations (statements) are equivalent:
  \[
    8 = 2^{3} \quad \Leftrightarrow \quad 3 = \log_{2}(8).
  \]
\end{example}
What this means is that a logarithm of a value is an exponent, or we
understand logarithms in terms of their equivalent exponential. Thus
the expression, \(\log_{2}(32)\), can be turned into a question by
writing it as an equation with an unknown, \(x\):
\[
  \log_{2}(32) \qquad \longrightarrow \qquad \log_{2}(32) = x.
\]
This equation is equivalent to asking, ``Which
exponent of 2 equals 32?''.
\[
  \log_{2}(32) = x \quad \Leftrightarrow \quad 32 = 2^{x}
\]
and of course the answer is 5, thus \(\log_{2}(32) = 5\).

Another way of stating the above equivalence is to think in terms of
function composition. Recall that when a function, \(f(x)\), and its
inverse, \(f^{-1}(x)\) are composed via function composition, \ie{}
\(\circ\), then the composition is the \emph{identity} function which
just maps the input to the output. Symbolically, this means:
\begin{align*}
  \log_{b}(x) \circ b^{x} &= \id(x) & \log_{b}(b^{x}) &= x \\
  b^{x} \circ \log_{b}(x) &= \id(x) & b^{\log_{b}(x)} &= x.
\end{align*}

\subsection{Properties of Logarithms}%
\label{sub:properties-logarithms}


There are a few useful properties of the logarithm function we should
review:
\begin{align}  % TODO label equations
  \log(a\cdot b) &= \log(a) + \log(b) \label{eq:log-product}\\
  \log\left( \frac{a}{b} \right) &= \log(a) - \log(b) \label{eq:log-ratio}\\
  \log\left( a^{n} \right) &= n \log(a) \label{eq:log-power}
\end{align}

The easiest way to remember
equations~\eqref{eq:log-product},\eqref{eq:log-ratio}
and~\eqref{eq:log-power} is to remember the following phrases:
\begin{quote}
  ``The log of a product is the sum of the logs.''\\
  ``The log of a quotient is the difference of the logs.''\\
  ``The log of an exponential is the exponent times the log''
\end{quote}
We will use the last property frequently when solving problems
involving exponential growth and decay.

\newpage

\begin{exercise}
  Rewrite the expression \(\log_{3}(5)+\log_{3}(8)-\log_{3}(2)\) as a
  single logarithm.

  \vspace*{\stretch{1}}
\end{exercise}

One final useful logarithm formula is the \textbf{change of
  base} formula which allows us to compute logarithms in any given
base, \(b\), via either \emph{common logarithms}, \ie{} base 10, or
\emph{natural logarithms} (base \(e \approx 2.718\)).

\begin{equation}
  \label{eq:logarithmic-base-change}
  \boxed{\log_{b}(a)  = \frac{\log(a)}{\log(b)} = \frac{\ln(a)}{\ln(b)}}
\end{equation}
\begin{example}
  Suppose you need to know what power of 5 equals 100, \ie{}
  \(5^{x} = 100\). The correct value for \(x\) must be between 2 and three because:
  \[
    5^{2} = 25 < 100 < 125 = 5^{3}
  \]
  We can solve this problem exactly by translating the exponential
  equation into the equivalent logarithmic equation,
  \[
    5^{x} = 100 \quad \Leftrightarrow \quad x = \log_{5}(100)
  \]
  and then using the change of base formula:
  \[\log_{5}(100)
    = \frac{\log(100)}{\log(5)}
    = \frac{\ln(100)}{\ln(5)}
    \approx 2.8614  \quad \Leftrightarrow \quad
    5^{2.8614} \approx 100.\]
\end{example}

\begin{exercise}
  Suppose a Pacific island was discovered by Polynesians around 400
  C.E. by 60 people. Assume the the population grows at a constant
  rate of 1\% per year.

  \begin{enumerate}
  \item Find an explicit formula to model this population.
    \vspace*{\stretch{1}}
  \item How long will it take for this population to reach 300,000
    people? What year is this? \vspace*{\stretch{1}}
  \item How long will it take for this population to reach 3 million
    people? What year? \vspace*{\stretch{1}}
  \end{enumerate}
\end{exercise}

\newpage

\begin{exercise}
  Carbon-14 is a radioactive isotope of carbon with a half-life of
  about 5730 years. What is the yearly decay rate as a percentage?

  \vspace*{\stretch{1}}
\end{exercise}

\begin{exercise}
  If you start with a 5 micro gram sample of pure carbon-14, how much
  of it will still be carbon-14 after 100 years?

  \vspace*{\stretch{1}}
\end{exercise}

\newpage

\section{Logistic Growth}%
\label{sec:logistic-growth}

Exponential growth has a serious problem because it predicts
``runaway'' growth after a certain amount of time. For example, if we
model world population via the exponential growth model, it predicts
that the earth will eventually reach populations of 100 billion, a
trillion and even much higher levels. Now, no one knows exactly how
many humans could actually live on Earth and with new technologies
that number might substantially increase, but unbounded growth cannot
last forever. At some point, people would be standing on top of other
people, not to mention valuable resources such as food and water would
eventually run out.

The fundamental assumption of the exponential growth model is that the
growth rate, \(r\), is constant. However, what social scientists have
learned about populations is that they do not grow forever in
unbounded fashion. Eventually, a population's growth rate will tend to
decrease as competition for valuable resources such as food and water
increases. The \textbf{logistic growth} model takes this variable
growth rate into account by assuming that the growth rate decreases
linearly with population size. That is,

\begin{align*}
  \label{eq:logistic-growth-rate}
  r_{L} &= r \left( 1 - \frac{P}{K} \right) \\ \\
  r &= \text{ base growth rate} \\
  P &= \text{ population} \\
  K &= \text{ carrying capacity}
\end{align*}

If we replace \(r\) in the recursive form of the exponential growth
model with \(r_{L}\), then we get:
\begin{equation}
  \label{eq:logistic-growth-model}
  P(x+1) = P(x) + r \left( 1 - \frac{P}{K} \right) P(x).
\end{equation}
The above equation is the recursive \textbf{logistic growth} model
\newpage

\begin{exercise}
  Suppose some tiny tropical fish are introduced into a large aquarium
  and the population grows at 7\% annually. A few years later, the
  caretaker notices that the population is not growing as fast as it
  did before.

  In what follows, assume the tank can sustain a maximum population of
  600 fish.

  \begin{enumerate}
  \item If there are currently 400 fish, use the logistic model to
    predict next year's population.

    \vspace*{\stretch{1}}


  \item What is the current rate of growth?

    \vspace*{\stretch{1}}

  \item What do you expect to happen to the growth rate as the
    population grows?

    \vspace*{\stretch{1}}
  \end{enumerate}
\end{exercise}

\begin{exercise}
  Suppose the world's human population has a base rate of growth of
  roughly 2.8\% and the earth has a carrying capacity of around 12
  billion humans.

  \begin{enumerate}
  \item Estimate the rate of growth in 1975 when the world population
    was 4 billion.

    \vspace*{\stretch{1}}

  \item Estimate the rate of growth in 1987 when the world population
    was 5 billion.

    \vspace*{\stretch{1}}

  \item Estimate the rate of growth in 2011 when the world population
    reached 7 billion.
  \end{enumerate}
\end{exercise}


%%% Local Variables:
%%% mode: latex
%%% TeX-master: "Notes"
%%% End:
