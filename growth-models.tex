\chapter{Growth Models}%
\label{chap:growth-models}

\begin{definition}
  A \textbf{recursive} function is a function defined in terms of its
  previous value(s).
\end{definition}

\begin{note}
  A recursive function is easy to define, but hard to evaluate because
  you have to compute all the previous values.
\end{note}

\begin{example}[Fibonacci Sequence]
  The famous Fibonacci function is defined recursively as follows:
  \[
    F(x+2) = F(x)+F(x+1).
  \]
  Defining \(F(0)=0\) and \(F(1)=1\) yields the following sequence of
  outputs:
  \begin{align*}
    F(0) &= 0 \\
    F(1) &= 1 \\
    F(2) &= F(0) + F(1) = 0+1 = 1 \\
    F(3) &= F(1) + F(2) = 1+1 = 2 \\
    F(4) &= F(2) + F(3) = 1+2 = 3 \\
    F(5) &= F(3) + F(4) = 2+3 = 5 \\
    F(6) &= F(4) + F(5) = 3+5 = 8 \\
    F(7) &= F(5) + F(6) = 5+8 = 13 \\
     & \: \: \: \vdots
  \end{align*}
  Each number in the sequence, is the sum of the previous two numbers:
  \(0, 1, 1, 2, 3, 5, 8, 13, 21, 34\ldots\)
\end{example}

\section{Linear Growth}%
\label{sec:linear-growth}

\subsection{Recursive Linear Growth Model}%
\label{sec:recursive-linear}

Let \(P=\) population, \(x=\) unit of time, then we can define next
years population, \(P(x+1)\) in terms of this years population,
\(P(x)\), plus a fixed amount, \(d\).

\begin{equation}
  \boxed{P(x+1) = P(x) + d}\label{eq:recursive-linear-model}
\end{equation}

Thus \textbf{linear growth} (or decay) is characterized by the
\emph{difference} between any two successive time periods being
constant: \(d\).
\[
  P(x+1) - P(x) = d
\]

\subsection{Explicit Linear Growth Model}%
\label{sec:explicit-linear}

Let \(P_0\equiv P(0)\), and recall the recursive definition:
\(P(x+1) = P(x) + d \)
\begin{align*}
  P(1) &= P_0 + d \\
  P(2) &= P(1) + d = (P_0 + d) + d = P_0 + 2d \\
  P(3) &= P(2) + d = (P_0 + 2d) + d = P_0 + 3d \\
  P(4) &= P(3) + d = (P_0 + 3d) + d = P_0 + 4d \\
       & \: \: \vdots \\
  P(x) &= P_0 + xd\\
  \intertext{Or equivalently,}
  \Aboxed{P(x) &= dx + P_0}
\end{align*}
where \(d\) is the slope or growth rate, and \(P_0\) is the
\(y\)--intercept, or initial population.

\section{Exponential Growth}%
\label{sec:exponential-growth}

\subsection{Recursive Exponential Growth Model}%
\label{sec:recursive-exponential}


\subsection{Explicit Exponential Growth Model}%
\label{sec:explicit-exponential}

\subsection{Logarithms}%
\label{sec:logarithms}


\section{Logistic Growth}%
\label{sec:logistic-growth}

\section{Summary}%
\label{sec:growth-models-summary}






%%% Local Variables:
%%% mode: latex
%%% TeX-master: "Notes"
%%% End:
