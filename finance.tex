\chapter{Finance}%
\label{chap:finance}

\section{Simple Interest}%
\label{sec:simple-interest}


\begin{definition}
  \textbf{Simple interest} refers to a one-time interest payment.
  Simple interest is used for short-term loans.
  \begin{align}
    I &= rP_{0} \notag \\
    A &= P_{0} + I \notag \\
    A &= P_{0} + rP_{0} \notag \\
    \Aboxed{A &= P_{0}(1+r)} \label{eq:simple-interest}
    \intertext{where,}
    I &= \text{ interest} \notag \\
    A &= \text{ account balance (amount in the account)} \notag \\
    P_{0} &= \text{ principal (starting amount)} \notag \\
    r &= \text{ interest rate (in decimal)} \notag
  \end{align}
\end{definition}

\begin{exercise}
  A friend agrees to loan you \$500 to help you buy a new laptop. You
  agree to pay them back in 30 days with 3\% interest. How much will
  you owe your friend in 30 days?

  \vspace*{\stretch{1}}
\end{exercise}

\newpage
\section{Simple Interest Over Time}%
\label{sec:simple-interest}


\begin{definition}
  \textbf{Simple interest over time} refers to a fixed interest
  payment that accrues once each year. Simple interest is used for
  bonds. A \textbf{bond} is a loan that you make to the government.
  All levels of government issue bonds; municipal, state and federal.

  \medskip

  \noindent When you purchase a bond, the government agrees to pay you a fixed
  interest payment each year until the bond \emph{matures} whereupon
  they return the initial amount they borrowed from you. A bond
  matures after a fixed number of years, often 5, 10, 20 or 30 years.

  \begin{align}
    I &= rP_{0}n \notag \\
    A_{n} &= P_{0} + I \notag \\
    A_{n} &= P_{0} + rP_{0}n \notag \\
    \Aboxed{A_{n} &= P_{0}(1+rn)} \label{eq:bond-interest}
    \intertext{where,}
    I &= \text{ interest} \notag \\
    A_{n} &= \text{ account balance (amount in the account)} \notag \\
    P_{0} &= \text{ principal (starting amount)} \notag \\
    r &= \text{ interest rate (in decimal)} \notag \\
    n &= \text{ time (in years)} \notag
  \end{align}
\end{definition}
\begin{note}
  Simple interest over time corresponds with \emph{linear growth}.
  \begin{align*}
    f(x) &= mx + b, \\
    A_{n} &= (rP_{0})n + P_{0},
  \end{align*}
  where \(rP_{0}\) is the slope (or fixed amount of yearly increase),
  and \(P_{0}\) is the \(y\)-intercept.
\end{note}

\begin{exercise}
  Suppose you buy a municipal bond from your local city government to
  help fund construction projects at the local zoo. The bond costs
  \$2000, pays 2.5\% interest annually and matures in 10 years. How
  much interest will you earn?

  \vspace*{\stretch{1}}
\end{exercise}

\newpage

\section{Compound Interest}%
\label{sec:compound-interest}

\begin{definition}
  \textbf{Compound interest} refers to any situation where prior
  interest payments also accrue interest.
  \begin{align}
    \Aboxed{A_{n}&= P_{0} {\left( 1+\frac{r}{k} \right)}^{nk}}
      \label{eq:compound-interest}\\
    \intertext{where,}
    A_{n} &= \text{ account balance after } n \text{ years.} \notag \\
    P_{0} &= \text{ principal (starting amount)} \notag \\
    r &= \text{ annual interest rate (in decimal)} \notag \\
    k &= \text{ the number of compounding periods per year} \notag
  \end{align}
\end{definition}
\begin{note}
  If \(k=1\), \ie{} there is only one compounding per year then the
  formula becomes
  \[
    \boxed{A_{n} = P_{0}{(1+r)}^{n},}
  \]
  which is the \emph{exponential population model}.
\end{note}

\begin{exercise}
  Suppose you deposit \$1000 into a bank account which earns 3\%
  interest compounded monthly. How much money will you have after one
  year?

  \vspace*{\stretch{1}}
  \noindent Answer: \$1,030.42
\end{exercise}

\newpage

\begin{exercise}
  A certificate of deposit (CD) is a savings instrument that many
  banks offer. It usually gives a higher interest rate, but you cannot
  access your investment for a specified length of time. Suppose you
  deposit \$3000 in a CD paying 6\% interest, compounded monthly. How
  much will you have in the account after 20 years?

  \vspace*{\stretch{1}}
  \noindent Answer: \$9,930.61
\end{exercise}

\begin{exercise}
  Again suppose you purchase a CD that pays 6\% interest compounded
  monthly. How long will it take for your initial deposit (the
  principal, \(P_{0}\)) to double?

  \medskip

  \noindent\emph{Hint: You don't need to know the principal, just use
    the symbol \(P_{0}\).}

  \vspace*{\stretch{1}}
\end{exercise}

\newpage

\section{Saving Annuities}%
\label{sec:annuities}

\begin{definition}
  A \textbf{savings annuity} is an account into which regular payments
  are made for a fixed number of years. When the annuity matures you
  receive the payments and the interest that each payment accrued.

  \begin{align}
    \Aboxed{A_{n}
    &= \frac{d\left[ {\left( 1+\frac{r}{k} \right)}^{nk} - 1 \right]}
      {\left( \frac{r}{k} \right)}} \\
    \intertext{where,}
    A_{n} &= \text{ account balance after } n \text{ years.} \notag \\
    d &= \text{ regular deposit} \notag \\
    r &= \text{ annual interest rate (in decimal)} \notag \\
    k &= \text{ the number of compounding periods per year} \notag
  \end{align}
\end{definition}

\begin{exercise}
  At the age of 30, Susan, opens a savings annuity account that pays
  6\% interest compounded monthly. She agrees to make a deposit of
  \$200 every month for 35 years. How much will the annuity be worth
  after 35 years when she is 65?

  \vspace*{\stretch{1}}
  \noindent Answer: \$284,000.06
\end{exercise}

\newpage

When retirement planning, the object is to determine how much you need
to save each month now in order to have a guaranteed fixed income
during your retirement years. This requires solving the savings
annuity formula for \(d\):

\begin{align}
  A_{n}
  &=\frac{d\left[ {\left( 1+\frac{r}{k} \right)}^{nk} - 1 \right]}
    {\left( \frac{r}{k} \right)} \notag \\
  A_{n}\left( \frac{r}{k} \right)
  &=d\left[ {\left( 1+\frac{r}{k} \right)}^{nk} - 1 \right] \notag \\
  \frac{A_{n}\left( \frac{r}{k} \right)}
  {\left[ {\left( 1+\frac{r}{k} \right)}^{nk} - 1 \right]}
  &=d \notag \\
  \Aboxed{d
  &=\frac
    {A_{n} \left( \frac{r}{k} \right)}
    {\left[ {\left( 1+\frac{r}{k} \right)}^{nk} - 1 \right]}}%
    \label{eq:savings-annuity-pmt}
\end{align}

\begin{exercise}
  Suppose you know that you are going to need \$400,000 when you
  retire. If you don't yet have any retirement savings, but you have
  35 years until you will retire, how much do you need to save each
  month into an annuity that pays 7\% interest compounded monthly to
  amass a \$400,000 nest egg?

  \vspace*{\stretch{1}}
  \noindent Answer: \$222.09
\end{exercise}

\newpage

\section{Payout Annuities}%
\label{sec:payout-annuities}

\begin{definition}
  A \textbf{payout annuity} is an account from which regular payments
  are made to you for a fixed number of years. After the fixed number
  of years, the annuity is worth nothing. Payout annuities are often
  used for retirement, payment of lottery winnings and payment of
  legal settlements.

  \begin{align}
    \Aboxed{P_{0}
    &= \frac{d\left[ {1 - \left( 1+\frac{r}{k} \right)}^{-nk} \right]}
      {\left( \frac{r}{k} \right)}} \label{eq:payout-annuity} \\
    \intertext{where,}
    P_{0} &= \text{Principal or starting balance} \notag \\
    d &= \text{ regular payment} \notag \\
    r &= \text{ annual interest rate (in decimal)} \notag \\
    k &= \text{ the number of compounding periods per year} \notag
  \end{align}
\end{definition}

\begin{exercise}
  In retirement, you determine that you will need \$2,500 per month
  for a total of 30 years. If the annuity earns 6\% interest, how much
  principal will you need to fund this retirement annuity? And how
  much will the annuity pay out over its lifetime?

  \vspace*{\stretch{1}}
  \noindent Answers: \$416,979.04, \$900,000.00
\end{exercise}

\newpage

\subsection{Retirement Planning}%
\label{sub:retirement-planning}

\begin{exercise}
  Suppose you are currently 25 years old, but you plan to begin saving
  for retirement at age 30 and you plan to retire at age 65. You
  determine that you would like to have \$4000 per month for 30 years.
  \begin{enumerate}
  \item How much principal will you need at age 65 to fund a payout
    annuity if the annuity yields 6\% interest?

    \vspace*{\stretch{3}}

  \item How much will you need to save each month beginning at age 30
    if your savings annuity yields 8\% interest?

    \vspace*{\stretch{3}}

  \item How much will you need to save each month if you begin saving
    for retirement at age 25 instead of 30, so you will have
    \(65-25=40\) total years of saving?

    \vspace*{\stretch{3}}

  \item How much money did you save for retirement? (Use calculation
    from question 3.)

    \vspace*{\stretch{1}}

  \item And how much money in total did you receive from that
    investment?

    \vspace*{\stretch{1}}

  \end{enumerate}

  \noindent Answers: (1) \$667,166.46, (2) \$290.85, (3) \$191.11,
  (4) \$91,732.80, (5) \$1,440,000.00
\end{exercise}

\newpage

\section{Loans}%
\label{sec:loans}

In this section, you will learn about conventional loans (also called
amortized loans or installment loans). Examples include auto loans and
home mortgages. These techniques do not apply to payday loans, add-on
loans, or other loan types where the interest is calculated up front.

One great thing about loans is that they use exactly the same formula
as a payout annuity. To see why, imagine that you had \$10,000
invested at a bank, and started taking out payments while earning
interest as part of a payout annuity, and after 5 years your balance
was zero. Flip that around, and imagine that you are acting as the
bank, and a car lender is acting as you. The car lender invests
\$10,000 in you. Since you're acting as the bank, you pay interest.
The car lender takes payments until the balance is zero.
\[
  \boxed{P_{0}
    = \frac{d\left[ {1 - \left( 1+\frac{r}{k} \right)}^{-nk} \right]}
    {\left( \frac{r}{k} \right)}}
\]
\begin{exercise}
  You can afford to pay \$280 per month for a car payment. If you
  qualify for an auto loan at 3\% interest for a term of 60 months (5
  years), how expensive of a car can you afford?

  \vspace*{\stretch{1}}
\end{exercise}

\noindent Answer: \$15,582.66

\newpage

Often when taking out a loan, you know the principal of the loan, but
you need to figure out what that amounts to in monthly payments. To do
this we can solve the payout annuity formula~\eqref{eq:payout-annuity}
for \(d\), the payment amount:

\begin{align*}
  P_{0}
  &= \frac{d\left[ {1 - \left( 1+\frac{r}{k} \right)}^{-nk} \right]}
    {\left( \frac{r}{k} \right)} \\
  P_{0}\left( \frac{r}{k} \right)
  &= d\left[ {1 - \left( 1+\frac{r}{k} \right)}^{-nk} \right] \\
  \frac{P_{0}\left( \frac{r}{k} \right)}
  {\left[ {1 - \left( 1+\frac{r}{k} \right)}^{-nk} \right]}
  &= d \\ \\
  \Aboxed{
  d &=
      \frac{P_{0}\left( \frac{r}{k} \right)}
      {\left[ {1 - \left( 1+\frac{r}{k} \right)}^{-nk} \right]}
  }
\end{align*}
\begin{exercise}
  You wish to take out a \$300,000 mortgage (home loan). If you
  qualify for a fixed interest rate of 4\% and you choose a 30 year
  term. How much will your monthly payments be?

  \vspace*{\stretch{1}}
\end{exercise}

\noindent Answer: \$1,432.25
\newpage

\section{Loan Payoff}%
\label{sec:loan-payoff}

To determine the remaining loan balance, we can think ``how much loan
will these loan payments be able to pay off in the remaining time on
the loan?''

\begin{note}
  Many people mistakenly believe that the payoff amount is simply
  their monthly payment times the number of remaining payments. This
  will always be more than the actual payoff amount.
\end{note}

\begin{exercise}
  Suppose you purchased your home with a 30 year mortgage for
  \$250,000 at a 3.5\% interest rate. Your monthly morgage payment is
  \$1122.61.

  \begin{enumerate}
  \item If you have made payments for 5 years (60 payments), what is
    the payoff amount? In other words how much money would you have to
    give the lender today to settle the mortgage debt?

    \vspace*{\stretch{3}}

  \item How much have you payed to the bank (mortgage lender) over
    five years?

    \vspace*{\stretch{1}}

  \item How much of the money you payed went towards interest?

    \vspace*{\stretch{1}}

  \end{enumerate}
\end{exercise}

\noindent Answers: (1) \$224,242.68, (2) \$67,356.60, (3) \$41,599.28

\newpage

\begin{exercise}
  Suppose your monthly car payment is \$366.

  \begin{enumerate}
  \item If your loan has a 5 year term and you have made payments for
    3 years (36 payments) what is your loan payoff amount?

    \vspace*{\stretch{1}}

  \item How much in total will you pay if you continue making payments
    for two years?

    \vspace*{\stretch{1}}

  \end{enumerate}
\end{exercise}


%%% Local Variables:
%%% mode: latex
%%% TeX-master: "Notes"
%%% End:
