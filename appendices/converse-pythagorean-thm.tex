\chapter{Converse of the Pythagorean Theorem}%
\label{app:converse-pythagorean-thm}

This \href{http://jwilson.coe.uga.edu/EMAT6680/Brown/6690/ConPythagThm.htm}%
{proof} comes from Jim Wilson at the University of Georgia.

The proof uses a technique called \textbf{proof by contradiction}
which is a little tricky, thus it has been relegated to obscurity in
this appendix. The idea of proof by contradiction is to assume that
the statement you are trying to prove is false, and then to show that
this leads to a logical contradiction, thus the statement must be
true. The proof also uses isosceles triangles. An \textbf{isosceles}
triangle has two sides that are the same length and two angles that
are equal.

\begin{proof}
  Suppose the triangle is \emph{not} a right triangle. Label the
  vertices \(A, B\) and \(C\) as pictured. There are two possibilities
  for the measure of angle \(C\): less than \(90^{\circ}\) (left
  picture) or greater than \(90^{\circ}\) (right picture).
  \begin{center}
    \includestandalone[scale=0.75]{figures/converse-pythagorean-thm-1}
  \end{center}
  Construct a perpendicular line segment \(CD\) as pictured below.
  \begin{center}
    \includestandalone[scale=0.75]{figures/converse-pythagorean-thm-2}
  \end{center}
  By the Pythagorean theorem, \(BD^{2} = a^{2} + b^{2} = c^{2}\), and
  so \(BD = c\). Thus we have isosceles triangles \(ACD\) and \(ABD\).
  It follows that we have congruent angles \(CDA = CAD\) and
  \(BDA = DAB\). But this contradicts the apparent inequalities (see
  picture) \(BDA < CDA = CAD < DAB\) (left picture) or
  \(DAB < CAD = CDA < BDA\) (right picture).
\end{proof}

%%% Local Variables:
%%% mode: latex
%%% TeX-master: "../Notes"
%%% End:
